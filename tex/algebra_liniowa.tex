\documentclass{article}

\usepackage{polski}
\usepackage[utf8]{inputenc}
\usepackage{amsthm}
\usepackage{amssymb}
\usepackage{amsmath}
\usepackage{booktabs}
\usepackage{dsfont}
\usepackage{pgfplots}

\pgfplotsset{compat=1.13}

\newcommand*\conj[1]{\bar{#1}}

\newtheorem{definition}{Definicja}[section]
\newtheorem{theorem}{Twierdzenie}[section]

\title{Notatki: Algebra liniowa}
\author{Adrian Startek}

\begin{document}
    \pagenumbering{gobble}
    \maketitle
    \newpage
    \pagenumbering{arabic}

    \section{Działanie, grupa, ciało}
    \begin{definition}
        Niech G będzie dowolnym zbiorem. \textbf{Działaniem} (dwuargumentowym) w zbiorze G nazywamy dowolne odwzorowanie $f: G \times G \rightarrow G$.
    \end{definition}

    \begin{definition}
        Zbiór G z określonym działaniem $\circ$ - parę $(\,G, \circ\,)$ nazwiemy grupą, jeśli spełnione są następujące warunki:
        \begin{enumerate}
            \item Działanie $\circ$ jest łączne: 
                \begin{equation*}
                    \forall_{a,b,c \in G} \, [\,(\,a \circ b\,) \circ c = a \circ (\,b \circ c\,)\,]
                \end{equation*}
            \item Istnieje element neutralny e:
                \begin{equation*}
                    \exists_{e \in G} \, \forall_{a \in G} \, (\,a \circ e = e \circ a = a\,)
                \end{equation*}
            \item Dla każdego elementu a istnieje element odwrotny $a^{-1}$:
                \begin{equation*}
                    \forall_{a \in G} \, \exists_{a^{-1} \in G} \, (\,a \circ a^{-1} = a^{-1} \circ a = e\,)
                \end{equation*}
        \end{enumerate}
        Jeżeli działanie $\circ$ jest dodatkowo przemienne, to $(\, G, \circ\,)$ nazwiemy \textbf{grupą przemienną} lub \textbf{Abelową} (Niels Henrik Abel - matematyk norweski). Opuszczając natomiast warunek 3 (istnienie elementu odwrotnego) otrzymamy definicję struktury ogólniejszej, zwanej \textbf{półgrupą}. 
    \end{definition}

    \begin{theorem}
        Jeśli $(\,G, \circ \,)$ jest grupą, to isteniej dokładnie 1 element neutralny.
    \end{theorem}

    \begin{proof}
        Załóżmy, że $e, e' \in G$ są elementami neutralnymi. Wtedy:
        \begin{equation*}
            e = e \circ e' = e' \circ e = e'
        \end{equation*}
        Co prowadzi do sprzeczności.
    \end{proof}

    \begin{theorem}
        Jeśli g i h są elementami grupy spełniającymi $g \circ h = e$, to są one wzajemnie odwrotne.
    \end{theorem}

    \begin{theorem}
        Jeśli $(\, G, \circ\, )$ jest grupą oraz $a \in G$ to istnieje dokładnie jeden element odwrotny $a^{-1}$.
    \end{theorem}

    \begin{definition}
        Zbiór G z określonymi działaniami "mnożenia" $\odot$ i "dodawania" $\oplus$ - trójkę $(\, G, \odot, \oplus\,)$ - nazywamy ciałem, jeżeli spełnione są warunki:
        \begin{enumerate}
            \item Oba działania są przemienne:
                \begin{equation*}
                    \forall_{a,b \in G} \, (\,a \oplus b = b \oplus a\,)
                \end{equation*}
                \begin{equation*}
                    \forall_{a,b \in G} \, (\,a \odot b = b \odot a\,)
                \end{equation*}
            \item Oba działania są łączne:
                \begin{equation*}
                    \forall_{a,b,c \in G} \, [\, (\,a \oplus b\,) \oplus c = a \oplus (\,b \oplus c\,)\,]
                \end{equation*}
                \begin{equation*}
                    \forall_{a,b,c \in G} \, [\, (\,a \odot b\,) \odot c = a \odot (\,b \odot c\,)\,]
                \end{equation*}
            \item Istnieje element neutralny dodawania ("zero" $\mathds{O}$) oraz element neutralny mnożenia ("jeden" $\mathds{1}$)
            \item Dla każdego elementu zbioru G istnieje element odwrotny względem dodawania:
                \begin{equation*}
                    \forall_{a \in G} \, (\, a \oplus a^{-1} = a^{-1} \oplus a = \mathds{O}\,)
                \end{equation*}
                Dla każdego elementu, poza elementem neutralnym dodawania, istnieje element odwrotny względem mnożenia:
                \begin{equation*}
                    \forall_{a \in G, a \neq \mathds{O}} \, (\, a \odot a^{-1} = a^{-1} \odot a = \mathds{1}\,)
                \end{equation*}
            \item Zachodzi rozdzielność mnożenia względem dodawania:
                \begin{equation*}
                    \forall_{a,b,c \in G} \, [\, a \odot (\, b \oplus c\,) = (\, a \odot b \,) \oplus (\, a \odot c\, )\,]
                \end{equation*}
            \item Elementy neutralne działań są od siebie różne:
                \begin{equation*}
                    \mathds{O} \neq \mathds{1}
                \end{equation*}
        \end{enumerate}
    \end{definition}
    \subsection{Liczby zespolone}
    \begin{definition}
        Niech $\mathds{C} := \mathds{R} \times \mathds{R}$. Określmy działania $\oplus ,\odot$:
        \begin{equation*}
            \oplus : (\,a,b\,)\oplus(\,c,d\,) = (\,a+c, b+d\,)
        \end{equation*}
        \begin{equation*}
            \odot : (\,a,b\,)\odot(\,c,d\,) = (\, ac-bd, ad+bc\,)
        \end{equation*}
        Trójkę $(\,G,\oplus,\odot\,)$ nazywamy ciałem liczb zespolonych.
    \end{definition}

    \paragraph{Postać kartezjańska.} Liczby zespolone posiadają naturalną interpretację geometryczną. Są one parami (uporządkowanymi) liczb rzeczywistych, więc można im przypisać punkty na płaszczyźnie. Liczbie $z = (\,a,b\,), z \in \mathds{C}$ odpowiada punkt o odciętej a i rzędnej b. Płaszczyznę, na której w ten sposób przedstawiamy liczby zespolone nazywamy \textbf{płaszczyzną Gaussa}.
    \begin{center}
        \begin{tikzpicture}
            \begin{axis}[
                    axis lines=middle,
                    axis equal, 
                    xmin=-4, 
                    ymin=-4, 
                    xmax=4, 
                    ymax=4,
                    xtick distance = 1,
                    ytick distance = 1
                ]
                \addplot+ [black, nodes near coords, only marks, point meta=explicit symbolic] table [meta=label] {
                    x   y   label
                    2   3   z=(2,3)
                };
                \draw [dotted] (2,3) -- (2,0);
                \draw [dotted] (0,3) -- (2,3);
            \end{axis}
        \end{tikzpicture}
    \end{center}
    \paragraph{Postać kanoniczna.} Podzbiór ciała $\mathds{C}$ złożony z liczb postaci $(\,x,0\,), x \in \mathds{R}$, również jest ciałem. Odwzorowanie $x \rightarrow (\,x,0\,)$ z $\mathds{R}$ w rozważany podzbiór $\mathds{C}$ zadaje izomorfizm ciał. Pozwala to na wprowadzenie utożsamienia $(\,x,0\,) \equiv x$. Wprowadźmy oznaczenie $i := (\,0,1\,)$ i nazwijmy ten element \textbf{jednostką urojoną}. Łatwo sprawdzić, że:
    \begin{equation*}
        i^2 = (\,-1,0\,) \equiv -1
    \end{equation*}
    Dowolną liczbę zespoloną $z = (\,a,b\,)$ można przedstawić w postaci:
    \begin{equation*}
        z = (\,a,b\,) = (\,a,0\,) + (\,b,0\,)(\,0,1\,) \equiv a+bi
    \end{equation*}
    Zapis liczby zespolonej z w postaci $z = a+bi$ nazywamy \textbf{postacią kanoniczną}. Liczbę $a in \mathds{R}$ nazywamy \textbf{częścią rzeczywistą} liczby zespolonej i oznaczamy $\Re{z}$ (lub $Re(z)$). Analogicznie, liczbę $b \in \mathds{R}$ nazywamy \textbf{częścią urojoną} i oznaczamy $\Im{z}$ (lub $Im(z)$).
    \begin{definition}
        \textbf{Liczbą sprzężoną} do liczby $z = a+bi$ nazywamy liczbę
        \begin{equation*}
            \conj{z} = a-bi
        \end{equation*}
    \end{definition}
    \begin{center}
        \begin{tikzpicture}
            \begin{axis}[
                    axis lines=middle,
                    axis equal, 
                    xmin=-4, 
                    ymin=-4, 
                    xmax=4, 
                    ymax=4,
                    xtick distance = 1,
                    ytick distance = 1
                ]
                \addplot+ [black, nodes near coords, only marks, point meta=explicit symbolic] table [meta=label] {
                    x   y   label
                    2   3   $z=(2,3)$
                    2   -3  $\conj{z}=(2,-3)$
                };
                \draw [dotted] (0,0) -- (2,3);
                \draw [dotted] (0,0) -- (2,-3);
            \end{axis}
        \end{tikzpicture}
    \end{center}
    W interpretacji geometrycznej liczba sprzężona $\conj{z}$ jest odbiciem liczby $z$ względem osi rzędnych.
    \paragraph{Postać trygonometryczna} Niech $z = a+bi$ będzie liczbą zespoloną.
    \begin{definition}
        \textbf{Modułem} liczby zespolonej nazywamy liczbę
        \begin{equation*}
            \mathopen|z\mathclose| = \sqrt{z\conj{z}} = \sqrt{a^2+b^2}
        \end{equation*}
    \end{definition}
    Moduł można interpretować jako długość odcinka pomiędzy początkiem układu współrzędnych a punktem reprezentującym liczbę zespoloną.
    \begin{center}
        \begin{tikzpicture}
            \begin{axis}[
                    axis lines=middle,
                    axis equal, 
                    xmin=-4, 
                    ymin=-4, 
                    xmax=4, 
                    ymax=4,
                    xtick = {0},
                    ytick = {0},
                    extra x ticks = {2},
                    extra x tick labels = {a},
                    extra y ticks = {3},
                    extra y tick labels = {b}
                ]
                \addplot+ [black, nodes near coords, only marks, point meta=explicit symbolic] table [meta=label] {
                    x   y   label
                    2   3   $(a,b)$
                };
                \addplot [blue, domain=0:2] {1.5*x}
                    node [pos=0.5, sloped, style={yshift=8pt}] {r}
                ;
                \draw [dotted] (0,3) -- (2,3);
                \draw [dotted] (2,0) -- (2,3);
                % \draw [solid] (0,0) -- (2,3);
            \end{axis}
        \end{tikzpicture}
    \end{center}
\end{document}
