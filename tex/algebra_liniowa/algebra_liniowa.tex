\section{Działanie, grupa, ciało}
\begin{definition}
    Niech G będzie dowolnym zbiorem. \textbf{Działaniem} (dwuargumentowym) w zbiorze G nazywamy dowolne odwzorowanie $f: G \times G \rightarrow G$.
\end{definition}

\begin{definition}
    Zbiór G z określonym działaniem $\circ$ - parę $(\,G, \circ\,)$ nazwiemy grupą, jeśli spełnione są następujące warunki:
    \begin{enumerate}
        \item Działanie $\circ$ jest łączne: 
            \begin{equation*}
                \forall_{a,b,c \in G} \, [\,(\,a \circ b\,) \circ c = a \circ (\,b \circ c\,)\,]
            \end{equation*}
        \item Istnieje element neutralny e:
            \begin{equation*}
                \exists_{e \in G} \, \forall_{a \in G} \, (\,a \circ e = e \circ a = a\,)
            \end{equation*}
        \item Dla każdego elementu a istnieje element odwrotny $a^{-1}$:
            \begin{equation*}
                \forall_{a \in G} \, \exists_{a^{-1} \in G} \, (\,a \circ a^{-1} = a^{-1} \circ a = e\,)
            \end{equation*}
    \end{enumerate}
    Jeżeli działanie $\circ$ jest dodatkowo przemienne, to $(\, G, \circ\,)$ nazwiemy \textbf{grupą przemienną} lub \textbf{Abelową} (Niels Henrik Abel - matematyk norweski). Opuszczając natomiast warunek 3 (istnienie elementu odwrotnego) otrzymamy definicję struktury ogólniejszej, zwanej \textbf{półgrupą}. 
\end{definition}

\begin{theorem}
    Jeśli $(\,G, \circ \,)$ jest grupą, to isteniej dokładnie 1 element neutralny.
\end{theorem}

\begin{proof}
    Załóżmy, że $e, e' \in G$ są elementami neutralnymi. Wtedy:
    \begin{equation*}
        e = e \circ e' = e' \circ e = e'
    \end{equation*}
    Co prowadzi do sprzeczności.
\end{proof}

\begin{theorem}
    Jeśli g i h są elementami grupy spełniającymi $g \circ h = e$, to są one wzajemnie odwrotne.
\end{theorem}

\begin{theorem}
    Jeśli $(\, G, \circ\, )$ jest grupą oraz $a \in G$ to istnieje dokładnie jeden element odwrotny $a^{-1}$.
\end{theorem}

\begin{definition}
    Zbiór G z określonymi działaniami "mnożenia" $\odot$ i "dodawania" $\oplus$ - trójkę $(\, G, \odot, \oplus\,)$ - nazywamy ciałem, jeżeli spełnione są warunki:
    \begin{enumerate}
        \item Oba działania są przemienne:
            \begin{equation*}
                \forall_{a,b \in G} \, (\,a \oplus b = b \oplus a\,)
            \end{equation*}
            \begin{equation*}
                \forall_{a,b \in G} \, (\,a \odot b = b \odot a\,)
            \end{equation*}
        \item Oba działania są łączne:
            \begin{equation*}
                \forall_{a,b,c \in G} \, [\, (\,a \oplus b\,) \oplus c = a \oplus (\,b \oplus c\,)\,]
            \end{equation*}
            \begin{equation*}
                \forall_{a,b,c \in G} \, [\, (\,a \odot b\,) \odot c = a \odot (\,b \odot c\,)\,]
            \end{equation*}
        \item Istnieje element neutralny dodawania ("zero" $\mathds{O}$) oraz element neutralny mnożenia ("jeden" $\mathds{1}$)
        \item Dla każdego elementu zbioru G istnieje element odwrotny względem dodawania:
            \begin{equation*}
                \forall_{a \in G} \, (\, a \oplus a^{-1} = a^{-1} \oplus a = \mathds{O}\,)
            \end{equation*}
            Dla każdego elementu, poza elementem neutralnym dodawania, istnieje element odwrotny względem mnożenia:
            \begin{equation*}
                \forall_{a \in G, a \neq \mathds{O}} \, (\, a \odot a^{-1} = a^{-1} \odot a = \mathds{1}\,)
            \end{equation*}
        \item Zachodzi rozdzielność mnożenia względem dodawania:
            \begin{equation*}
                \forall_{a,b,c \in G} \, [\, a \odot (\, b \oplus c\,) = (\, a \odot b \,) \oplus (\, a \odot c\, )\,]
            \end{equation*}
        \item Elementy neutralne działań są od siebie różne:
            \begin{equation*}
                \mathds{O} \neq \mathds{1}
            \end{equation*}
    \end{enumerate}
\end{definition}
\subsection{Liczby zespolone}
\begin{definition}
    Niech $\mathds{C} := \mathds{R} \times \mathds{R}$. Określmy działania $\oplus ,\odot$:
    \begin{equation*}
        \oplus : (\,a,b\,)\oplus(\,c,d\,) = (\,a+c, b+d\,)
    \end{equation*}
    \begin{equation*}
        \odot : (\,a,b\,)\odot(\,c,d\,) = (\, ac-bd, ad+bc\,)
    \end{equation*}
    Trójkę $(\,G,\oplus,\odot\,)$ nazywamy ciałem liczb zespolonych.
\end{definition}

\paragraph{Postać kartezjańska.} Liczby zespolone posiadają naturalną interpretację geometryczną. Są one parami (uporządkowanymi) liczb rzeczywistych, więc można im przypisać punkty na płaszczyźnie. Liczbie $z = (\,a,b\,), z \in \mathds{C}$ odpowiada punkt o odciętej a i rzędnej b (rysunek \ref{fig:lba_zespolona}). Płaszczyznę, na której w ten sposób przedstawiamy liczby zespolone nazywamy \textbf{płaszczyzną Gaussa}.
\begin{figure}[htbp!]
    \centering
    \caption{Liczba zespolona jako punkt na płaszczyźnie}
    \label{fig:lba_zespolona}
    \vspace{3mm}
    \begin{tikzpicture}
        \begin{axis}[
                axis lines=middle,
                axis equal, 
                xmin=-4, 
                ymin=-4, 
                xmax=4, 
                ymax=4,
                xtick distance = 1,
                ytick distance = 1
            ]
            \addplot+ [black, nodes near coords, only marks, point meta=explicit symbolic] table [meta=label] {
                x   y   label
                2   3   z=(2,3)
            };
            \draw [dotted] (2,3) -- (2,0);
            \draw [dotted] (0,3) -- (2,3);
        \end{axis}
    \end{tikzpicture}
\end{figure}
\paragraph{Postać kanoniczna.} Podzbiór ciała $\mathds{C}$ złożony z liczb postaci $(\,x,0\,), x \in \mathds{R}$, również jest ciałem. Odwzorowanie $x \rightarrow (\,x,0\,)$ z $\mathds{R}$ w rozważany podzbiór $\mathds{C}$ zadaje izomorfizm ciał. Pozwala to na wprowadzenie utożsamienia $(\,x,0\,) \equiv x$. Wprowadźmy oznaczenie $i := (\,0,1\,)$ i nazwijmy ten element \textbf{jednostką urojoną}. Łatwo sprawdzić, że:
\begin{equation*}
    i^2 = (\,-1,0\,) \equiv -1
\end{equation*}
Dowolną liczbę zespoloną $z = (\,a,b\,)$ można przedstawić w postaci:
\begin{equation*}
    z = (\,a,b\,) = (\,a,0\,) + (\,b,0\,)(\,0,1\,) \equiv a+bi
\end{equation*}
Zapis liczby zespolonej z w postaci $z = a+bi$ nazywamy \textbf{postacią kanoniczną}. Liczbę $a \in \mathds{R}$ nazywamy \textbf{częścią rzeczywistą} liczby zespolonej i oznaczamy $\Re{z}$ (lub $Re(z)$). Analogicznie, liczbę $b \in \mathds{R}$ nazywamy \textbf{częścią urojoną} i oznaczamy $\Im{z}$ (lub $Im(z)$).
\begin{definition}
    \textbf{Liczbą sprzężoną} do liczby $z = a+bi$ nazywamy liczbę
    \begin{equation*}
        \conj{z} = a-bi
    \end{equation*}
\end{definition}
\begin{figure}[htbp!]
    \centering
    \caption{Interpretacja geometryczna liczby sprzężonej}
    \label{fig:lba_sprzezona}
    \vspace{3mm}
    \begin{tikzpicture}
        \begin{axis}[
                axis lines=middle,
                axis equal, 
                xmin=-4, 
                ymin=-4, 
                xmax=4, 
                ymax=4,
                xtick distance = 1,
                ytick distance = 1
            ]
            \addplot+ [black, nodes near coords, only marks, point meta=explicit symbolic] table [meta=label] {
                x   y   label
                2   3   $z=(2,3)$
                2   -3  $\conj{z}=(2,-3)$
            };
            \draw [dotted] (0,0) -- (2,3);
            \draw [dotted] (0,0) -- (2,-3);
        \end{axis}
    \end{tikzpicture}
\end{figure}
W interpretacji geometrycznej liczba sprzężona $\conj{z}$ jest odbiciem liczby $z$ względem osi odciętych (rysunek \ref{fig:lba_sprzezona}).
\paragraph{Postać trygonometryczna.} Niech $z = a+bi$ będzie liczbą zespoloną.
\begin{definition}
    \textbf{Modułem} liczby zespolonej $z$ nazywamy liczbę
    \begin{equation*}
        r = \mathopen|z\mathclose| = \sqrt{z\conj{z}} = \sqrt{a^2+b^2}
    \end{equation*}
\end{definition}
Moduł można interpretować jako długość odcinka pomiędzy początkiem układu współrzędnych a punktem reprezentującym liczbę zespoloną (rysunek \ref{fig:modul_lby_zespolonej}).
\begin{figure}[tbp!]
    \centering
    \caption{Moduł liczby zespolonej}
    \label{fig:modul_lby_zespolonej}
    \vspace{3mm}
    \begin{tikzpicture}
        \begin{axis}[
                axis lines=middle,
                axis equal, 
                xmin=-4, 
                ymin=-4, 
                xmax=4, 
                ymax=4,
                xtick = {0},
                ytick = {0},
                extra x ticks = {2},
                extra x tick labels = {a},
                extra y ticks = {3},
                extra y tick labels = {b}
            ]
            \addplot+ [black, nodes near coords, only marks, point meta=explicit symbolic] table [meta=label] {
                x   y   label
                2   3   $(a,b)$
            };
            \addplot [blue, domain=0:2] {1.5*x}
                node [pos=0.5, sloped, style={yshift=8pt}] {r}
            ;
            \draw [dotted] (0,3) -- (2,3);
            \draw [dotted] (2,0) -- (2,3);
        \end{axis}
    \end{tikzpicture}
\end{figure}
Niech $\varphi$ będzie kątem pomiędzy dodatnią półosią rzeczywistą a odcinkiem łączącym początek układu współrzędnych a punktem reprezentującym liczbę zezpoloną. Zachodzi wtedy:
\begin{equation*}
    r = \mathopen|z\mathclose|
\end{equation*}
\begin{equation*}
    sin \varphi = \frac{b}{\mathopen|z\mathclose|} \Longrightarrow b = \mathopen|z\mathclose| sin \varphi
\end{equation*}
\begin{equation*}
    cos \varphi = \frac{a}{\mathopen|z\mathclose|} \Longrightarrow a = \mathopen|z\mathclose| cos \varphi
\end{equation*}
Wynika z tego możliwość przedstawienia liczby zespolonej $z = a + bi$ w postaci:
\begin{equation*}
    z = a + bi = \mathopen|z\mathclose| cos \varphi + i \mathopen|z\mathclose| sin \varphi = \mathopen|z\mathclose|(cos \varphi + i\, sin \varphi)
\end{equation*}
\begin{theorem}
    Niech $z \in \mathds{C}$, $z = \mathopen|z\mathclose|(cos \varphi + i\, sin \varphi)$. Zachodzi:
    \begin{equation*}
        \forall_{n \in \mathds{N}} \, z^n = {\mathopen|z\mathclose|}^n (cos(n\varphi) + i\, sin(n\varphi))
    \end{equation*}
\end{theorem}
\begin{theorem}
    Niech $z \in \mathds{C}$, $z \neq (0,0)$, $z = \mathopen|z\mathclose|(cos\varphi + i\, sin\varphi)$.
    Istnieje dokładnie $n$ pierwiastków $n$-tego stopnia z liczby $z$. $k$-ty pierwiastek dany jest wzorem:
    \begin{equation*}
        w_k = \sqrt[n]{\mathopen|z\mathclose|}\left(cos\frac{\varphi+2k\pi}{n} + i\, sin \frac{\varphi+2k\pi}{n}\right), k \in \langle0; n-1 \rangle, k \in \mathds{N}
    \end{equation*}
    \begin{proof}
        Niech $w = \mathopen|w\mathclose|(cos \alpha + i\, sin \alpha)$ będzie pierwiastkiem $n$-tego stopnia z liczby $z$. Zatem:
        \begin{equation*}
            w^n = z
        \end{equation*}
        \begin{equation*}
            w^n = {\mathopen|w\mathclose|}^n(cos\, n\alpha + i\, sin\, n\alpha) = \mathopen|z\mathclose|(cos\varphi + i\, sin\varphi)
        \end{equation*}
        \begin{equation*}
            {\mathopen|w\mathclose|}^n = \mathopen|z\mathclose| \Longrightarrow \mathopen|w\mathclose| = \sqrt[n]{\mathopen|z\mathclose|}
        \end{equation*}
        \begin{equation*}
            n\alpha = \varphi+2k\pi \Longrightarrow \alpha = \frac{\varphi+2k\pi}{n}
        \end{equation*}
    \end{proof}
\end{theorem}
\section{Przestrzenie liniowe}
\begin{definition}
    Niech będzie dane ciało K oraz niepusty zbiór V z określonymi dwoma działaniami:
    \begin{equation*}
        \tag{dodawanie}
        \oplus: V \times V \ni (\,x,y\,) \rightarrow x \oplus y \in V
    \end{equation*}
    \begin{equation*}
        \tag{mnożenie przez element ciała}
        \odot: K \times V \ni (\,\lambda,x\,) \rightarrow \lambda \odot x \in V
    \end{equation*}
    Strukturę $(\,V,K,\oplus,\odot)$ nazwiemy \textbf{przestrzenią wektorową} (lub \textbf{liniową}) nad ciałem K, jeśli spełnione są warunki:
    \begin{enumerate}
        \item $(\,V,\oplus\,)$ jest grupą abelową
        \item $\forall_{x \in V}$:
            \begin{equation*}
                1\odot x = x
            \end{equation*}
        \item $\forall_{x \in V}\,\forall_{\alpha,\beta \in K}$:
            \begin{equation*}
                \tag{prawo łączności}
                \alpha\odot(\beta\odot x) = (\alpha\beta)\odot x
            \end{equation*}
        \item $\forall_{x,y \in V}\, \forall_{\alpha,\beta \in K}$:
            \begin{equation*}
                \tag{rozdzielność dodawania względem mnożenia}
                (\alpha + \beta)\odot x = \alpha \odot x \oplus \beta \odot x
            \end{equation*}
        \item $\forall_{x,y \in V}\, \forall_{\alpha \in K}$:
            \begin{equation*}
                \tag{rozdzielność mnożenia względem dodawania}
                \alpha\odot(x \oplus y) = \alpha\odot x \oplus \alpha\odot y
            \end{equation*}
    \end{enumerate}
    Elementy zbioru V nazywamy \textbf{wektorami}, a ciała K \textbf{skalarami}.
\end{definition}
\begin{theorem}
    Jeśli V jest przestrzenią liniową nad K, to dla każdego $v \in V, \, \alpha \in K$ zachodzi:
    \begin{enumerate}
        \item $ 0\odot v = \mathds{O}$
        \item $\alpha \odot \mathds{O} = \mathds{O}$
        \item $\alpha \odot v = \mathds{O} \Longrightarrow \alpha = 0 \vee v = \mathds{O}$
    \end{enumerate}
    \begin{proof} Kolejno:
        \begin{enumerate}
            \item
                \begin{equation*}
                    1\odot v = (1 + 0) \odot v = 1\odot v \oplus 0 \odot v
                \end{equation*}
                \begin{equation*}
                    1\odot v - 1\odot v = (1\odot v - 1\odot v) \oplus 0 \odot v
                \end{equation*}
                \begin{equation*}
                    \mathds{O} = 0 \odot v
                \end{equation*}
            \item
                \begin{equation*}
                    \alpha\odot\mathds{O} = \alpha(\mathds{O}\oplus\mathds{O}) = \alpha\odot\mathds{O}\oplus\alpha\odot\mathds{O}
                \end{equation*}
                \begin{equation*}
                    \alpha\odot\mathds{O}-\alpha\odot\mathds{O} = (\alpha\odot\mathds{O}-\alpha\odot\mathds{O})\oplus\alpha\odot\mathds{O}
                \end{equation*}
                \begin{equation*}
                    \mathds{O} = \alpha\odot\mathds{O}
                \end{equation*}
            \item
                \begin{equation*}
                    \alpha\odot v = \mathds{O}
                \end{equation*}
                Wystarczy pokazać, że jeśli  $\alpha\neq 0$, to $v = \mathds{O}$
                \begin{equation*}
                    \exists_{\alpha^{-1} \in K}\, \alpha\alpha^{-1} = 1
                \end{equation*}
                \begin{equation*}
                    \alpha\odot v = \mathds{O}
                \end{equation*}
                \begin{equation*}
                    \alpha\alpha^{-1}\odot v = \alpha^{-1}\odot\mathds{O}
                \end{equation*}
                \begin{equation*}
                    1\odot v = \mathds{O}
                \end{equation*}
                \begin{equation*}
                    v = \mathds{O}
                \end{equation*}
        \end{enumerate}
    \end{proof}
\end{theorem}
\begin{definition}
    Niech V będzie przestrzenią liniową nad ciałem K. Podzbiór $W \subset V$ nazywamy \textbf{podprzestrzenią liniową} przestrzeni V, jeśli W również jest przestrzenią liniową nad ciałem K.
\end{definition}
\begin{theorem}
    Podzbiór $W \subset V$ jest podprzestrzenią, wtedy i tylko wtedy, gdy zachodzi:
    \begin{equation*}
        \forall_{v_1,v_2 \in W} \, v_1 \oplus v_2 \in W
    \end{equation*}
    \begin{equation*}
        \forall_{\alpha \in K, v \in W} \, \alpha\odot v \in W
    \end{equation*}
\end{theorem}
\begin{theorem}
    \label{th:czesc_wspolna_podprzestrzeni}
    Niech V będzie przestrzenią liniową nad ciałem K, oraz niech $\{W_s\}$ będzie rodziną podprzestrzeni V. Wtedy:
    \begin{equation*}
        W = \bigcap_s W_s
    \end{equation*}
    jest podprzestrzenią liniową V.
        \begin{proof}
            Niech $v_1,v_2 \in W$
            \begin{equation*}
            v_1,v_2 \in W \Longrightarrow v_1,v_2 \in \bigcap_s W_s \Longrightarrow \forall_s v_1,v_2 \in W_s \Longrightarrow \forall_s v_1 \oplus v_2 \in W_s 
            \end{equation*}
            \begin{equation*}
                \Longrightarrow v_1\oplus v_2 \in \bigcap_s W_s \Leftrightarrow v_1\oplus v_2 \in W
            \end{equation*}
            Niech $v \in W, \alpha \in K$
            \begin{equation*}
                v \in W \Longrightarrow \forall_s v \in W_s \Longrightarrow \forall_s \alpha\odot v \in W_s \Longrightarrow \alpha\odot v \in W
            \end{equation*}
        \end{proof}
\end{theorem}
\begin{definition}
    Niech V będzie przestrzenią liniową nad ciałem K. Weźmy układ wektorów $\{v_1, \ldots, v_n\} \subset V$. Wektor
    \begin{equation*}
        v = \alpha_1 v_1 + \alpha_2 v_2 + \ldots + \alpha_n v_n
    \end{equation*}
    nazywamy \textbf{kombinacją liniową} wektorów $v_1,\ldots,v_n$.
\end{definition}
\begin{definition}
    \textbf{Powłoką liniową} podzbioru $M \subset V$ (gdzie V jest przestrzenią liniową) nazywamy zbiór wszystkich kombinacji liniowych wektorów z M i oznaczamy L(M).
\end{definition}
\begin{theorem}
    Niech V będzie przestrzenią liniową nad ciałem K i $M \subset V$ będzie podprzestrzenią. Powłoka liniowa L(M) jest podprzestrzenią liniową V.
    \begin{proof}
        \begin{equation*}
            v_1,v_2 \in L(M) \Longrightarrow
        \end{equation*}
        \begin{equation*}
            v_1 = \alpha_1 w_1 + \ldots + \alpha_n w_n, \, w_i \in M
        \end{equation*}
        \begin{equation*}
            v_2 = \beta_1 u_1 + \ldots + \beta_n u_n, \, u_i \in M
        \end{equation*}
        \begin{equation*}
            v_1 + v_2 = \alpha_1 w_1 + \ldots + \alpha_n w_n + \beta_1 u_1 \ldots \beta_n u_n 
        \end{equation*}
        zatem $v_1 + v_2$ jest kombinacją liniową wektorów $w_1,\ldots,w_n,u_1,\ldots,u_n \in M$, czyli $v_1 + v_2 \in M$.\\
        Niech $\alpha \in K, v \in L(M), v = \alpha_1 v_1+\ldots +\alpha_n v_n$
        \begin{equation*}
            \alpha v = \alpha(\alpha_1 v_1+\ldots +\alpha_n v_n) = \alpha\alpha_1 v_1+\ldots + \alpha\alpha_n v_n \in L(M)
        \end{equation*}
    \end{proof}
\end{theorem}
\begin{theorem}
    L(M) jest najmniejszą (w sensie zawierania zbiorów) podprzestrzenią V zawierającą M (jeżeli $W \subset V$ jest podprzestrzenią liniową V, taką że $M \subset W$, to $L(M) \subset W$).
    \begin{proof}
        Weźmy zbiór wszystkich podprzestrzeni liniowych zawierających M:
        \begin{equation*}
            \{W_s: s \in S\}
        \end{equation*}
        Niech
        \begin{equation*}
            \overline{W} = \bigcap_s W_s
        \end{equation*}
        Z twierdzenia \ref{th:czesc_wspolna_podprzestrzeni} $\overline{W}$ jest podprzestrzenią. Zauważmy: $\forall_s \overline{W} \subset W_s$. Pokażmy, że $\overline{W} = L(M)$, czyli $\overline{W} \subset L(M) \wedge L(M) \subset \overline{W}$. 
        \begin{enumerate}
            \item $\overline{W} \subset L(M)$. Skoro $\{W_s\}$ zawiera wszystkie podprzestrzenie zawierające M, to:
                \begin{equation*}
                    \exists_{s_0 \in S} W_{s_0} = L(M)
                \end{equation*}
            \item $L(M) \subset \overline{W}$. Niech $v \in L(M)$, wtedy:
                \begin{equation*}
                    v = \alpha_1 v_1 + \ldots + \alpha_n v_n, \, v_i \in M
                \end{equation*}
                \begin{equation*}
                    \forall_s \, \alpha_1 v_1 + \ldots +\alpha_n v_n \in W_s \Longrightarrow \forall_s \, \alpha_1 v_1 + \ldots +\alpha_n v_n \in \bigcap_s W_s \Longrightarrow v \in \overline{W}
                \end{equation*}
        \end{enumerate}
    \end{proof}
\end{theorem}
