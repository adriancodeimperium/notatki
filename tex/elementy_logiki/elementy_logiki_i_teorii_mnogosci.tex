\section{Podstawowe definicje i oznaczenia}
    \begin{definition}
        \index{zdanie logiczne}
        Zdanie logiczne: zdanie, któremu można przyporządkować wartość logiczną "prawda"(1) lub "fałsz"(0).
    \end{definition}
    \begin{definition}
        \index{tautologia}
        Tautologia: zdanie logiczne, które zawsze jest prawdziwe.
    \end{definition}
    \begin{definition}
        \index{funckcja!zdaniowa}
        Funkcja zdaniowa $\phi(x)$: wyrażenie, które po podstawieniu konkretnej wartości $x$ staje się zdaniem logicznym.
    \end{definition}
    \begin{definition}
        \index{para uporządkowana}
        Para uporządkowana (x,y): zbiór $\{\{x\},\{x,y\}\}$. Elementem pierwszym w parze jest ten, który jest elementem obu zbiorów, co jednoznacznie określa kolejność.
    \end{definition}

    \subsection{Zbiory liczbowe}
        $\mathbb{N}$ - zbiór liczb naturalnych\\
        $\mathbb{Z}$ - zbiór liczb całkowitych\\
        $\mathbb{Q}$ - zbiór liczb wymiernych\\
        $\mathbb{R}$ - zbiór liczb rzeczwistych

        \paragraph{} Fakt należenia elementu x do zbioru A oznacza się przez $x \in A$. Analogicznie, "x nie należy do zbioru A" ocznacza się $x \notin A$.
        \paragraph{} Zbiór można definiować podając jego elementy wprost: $A = \{a, b, c\}$ lub zadając warunek na przynależność elementów do zbioru: $A = \{x \in X: \phi (x)\}$. 

    \subsection{Kwantyfikatory}
        \paragraph{Kwantyfikator ogólny.} Wyrażenie "dla każdego x należącego do X zachodzi $\phi(x)$" oznacza się: $\forall_{x \in X} \, \phi(x)$
        \paragraph{Kwantyfikator szczególny.} Wyrażenie "istnieje x należący do X, dla którego zachodzi $\phi(x)$" oznacza się: $\exists_{x \in X} \, \phi(x)$
        \paragraph{Zaprzeczenia kwantyfikatorów.} Zachodzi:
        \begin{equation*}
            \neg [\,\forall_{x \in \mathbb{R}} \, \phi(x)\,] \Leftrightarrow \exists_{x \in \mathbb{R}} \, \neg \phi(x)
        \end{equation*}
        \begin{equation*}
            \neg [\,\exists_{x \in \mathbb{R}} \, \phi(x)\,] \Leftrightarrow \forall_{x \in \mathbb{R}} \, \neg \phi(x)
        \end{equation*}
        \begin{equation*}
            \neg [\,\forall_{x \in \mathbb{R}} \, \phi(x) \vee \psi(x)\,] \Leftrightarrow \exists_{x \in \mathbb{R}} \, \neg \phi(x) \wedge \neg \psi(x)
        \end{equation*}
        \begin{equation*}
            \neg [\,\exists_{x \in \mathbb{R}} \, \phi(x) \wedge \psi(x)\,] \Leftrightarrow \forall_{x \in \mathbb{R}} \, \neg \phi(x) \vee \neg \psi(x)
        \end{equation*}

\section{Rachunek zdań logicznych}
    \subsection{Ważniejsze operacje na zdaniach}
        \paragraph{Negacja}
        Wartością negacji zdania logicznego jest wartość odwrotna do wartości tego zdania (tabela \ref{tab:negacja}).
        \begin{table}[htbp!]
            \centering
            \caption{Negacja}
            \label{tab:negacja}
            \vspace{3mm}
            \begin{tabular}{cc}
                \textbf{$p$} & \textbf{$\neg p$} \\
                \midrule
                0 & 1\\
                1 & 0 \\
                \bottomrule
            \end{tabular}
        \end{table}

        \paragraph{Alternatywa}
        Alternatywa przyjmuje wartość "prawda", jeśli co najmniej jedno ze zdań jest prawdziwe (tabela \ref{tab:alternatywa}).
        \begin{table}[htbp!]
            \centering
            \caption{Alternatywa}
            \label{tab:alternatywa}
            \vspace{3mm}
            \begin{tabular}{ccc}
                \textbf{$p$} & \textbf{$q$} & \textbf{$p \vee q$} \\
                \midrule
                0 & 0 & 0\\
                0 & 1 & 1\\
                1 & 0 & 1\\
                1 & 1 & 1\\
                \bottomrule
            \end{tabular}
        \end{table}

        \paragraph{Koniunkcja}
        Koniunkcja przyjmuje wartość "prawda", tylko jeśli oba zdania są prawdziwe (tabela \ref{tab:koniunkcja}).
        \begin{table}[htbp!]
            \centering
            \caption{Koniunkcja}
            \label{tab:koniunkcja}
            \vspace{3mm}
            \begin{tabular}{ccc}
                \textbf{$p$} & \textbf{$q$} & \textbf{$p \wedge q$} \\
                \midrule
                0 & 0 & 0\\
                0 & 1 & 0\\
                1 & 0 & 0\\
                1 & 1 & 1\\
                \bottomrule
            \end{tabular}
        \end{table}

        \paragraph{Implikacja}
        Implikacja ($p \Longrightarrow q$) jest prawdziwa, jeśli zarówno poprzednik ($p$) jak i następnik ($q$) są prawdziwe lub \textbf{poprzednik jest fałszywy (z fałszu wynika wszystko)}. (tabela \ref{tab:implikacja})
        \begin{table}[htbp!]
            \centering
            \caption{Implikacja}
            \label{tab:implikacja}
            \vspace{3mm}
            \begin{tabular}{ccc}
                \textbf{$p$} & \textbf{$q$} & \textbf{$p \Longrightarrow q$} \\
                \midrule
                0 & 0 & 1\\
                0 & 1 & 1\\
                1 & 0 & 0\\
                1 & 1 & 1\\
                \bottomrule
            \end{tabular}
        \end{table}
        
        \paragraph{Równoważność}
        Równoważność przyjmuje wartość "prawda" jeśli oba zdania mają tą samą wartość (tabela \ref{tab:rownowaznosc}).
        \begin{table}[htbp!]
            \centering
            \caption{Równoważność}
            \label{tab:rownowaznosc}
            \vspace{3mm}
            \begin{tabular}{ccc}
                \textbf{$p$} & \textbf{$q$} & \textbf{$p \Leftrightarrow q$} \\
                \midrule
                0 & 0 & 1\\
                0 & 1 & 0\\
                1 & 0 & 0\\
                1 & 1 & 1\\
                \bottomrule
            \end{tabular}
        \end{table}
        
        \paragraph{Kreska Sheffera (NAND)}
        Zaprzeczenie koniunkcji (tabela \ref{tab:nand}).
        \begin{table}[htbp!]
            \centering
            \caption{NAND}
            \label{tab:nand}
            \vspace{3mm}
            \begin{tabular}{ccc}
                \textbf{$p$} & \textbf{$q$} & \textbf{$p \mid q$} \\
                \midrule
                0 & 0 & 1\\
                0 & 1 & 1\\
                1 & 0 & 1\\
                1 & 1 & 0\\
                \bottomrule
            \end{tabular}
        \end{table}
        
        \paragraph{NOR}
        Zaprzeczenie alternatywy (tabela \ref{tab:nor}). 
        \begin{table}[htbp!]
            \centering
                \caption{NOR}
                \label{tab:nor}
                \vspace{3mm}
                \begin{tabular}{ccc}
                    \textbf{$p$} & \textbf{$q$} & \textbf{$p\downarrow q$} \\
                    \midrule
                    0 & 0 & 1\\
                    0 & 1 & 0\\
                    1 & 0 & 0\\
                    1 & 1 & 0\\
                    \bottomrule
                \end{tabular}
        \end{table}
        
        \begin{theorem}
            Za pomocą \textbf{NAND} lub \textbf{NOR} można wyrazić wszystkie inne funktory.
        \end{theorem}
    
    \subsection{Ważniejsze tautologie}
        \begin{equation*}
            \tag{prawo tożsamości}
            p \Longrightarrow p
        \end{equation*}
        \begin{equation*}
            \tag{prawo symplifikacji}
            p \Longrightarrow (q \Longrightarrow p)
        \end{equation*}
        \begin{equation*}
            \tag{prawo podwójnej negacji}
            p \Leftrightarrow \neg(\neg p)
        \end{equation*}
        \begin{equation*}
            \tag{prawo wyłączonego środka}
            p \vee \neg p\end
        {equation*}
        \begin{equation*}
            (\neg p \Longrightarrow p) \Longrightarrow p
        \end{equation*}
        \begin{equation*}
            \tag{prawo Dunsa Szkota}
            \neg p \Longrightarrow (p \Longrightarrow q)
        \end{equation*}
        \begin{equation*}
            \tag{prawo De Morgana}
            \neg (p \vee q) \Leftrightarrow (\neg p) \wedge (\neg q)
        \end{equation*}
        \begin{equation*}
            \tag{prawo De Morgana}
            \neg (p \wedge q) \Leftrightarrow (\neg p) \vee (\neg q)
        \end{equation*}
        \begin{equation*}
            \neg(p \Longrightarrow q) \Leftrightarrow p \wedge (\neg q)
        \end{equation*}
        \begin{equation*}
            \neg(p \Leftrightarrow q) \Leftrightarrow (p \wedge \neg q) \vee (\neg p \wedge q)
        \end{equation*}

\section{Rachunek zbiorów}
    \subsection{Operacje i zależności}
        \paragraph{Zawieranie.} Zbiór A zawiera się w zbiorze B (A jest podzbiorem B), ozn. $A \subset B$, jeśli każdy element zbioru A jest również elementem zbioru B:
        \begin{equation*}
            A \subset B \Leftrightarrow \forall_{x\in A} \,  x\in B
        \end{equation*}
        
        \paragraph{Równość.} Zbiory A i B są równe, jeśli są one nawzajem swoimi podzbiorami:
        \begin{equation*}
            A = B \Leftrightarrow A \subset B \wedge B \subset A
        \end{equation*}

        \paragraph{Działania na zbiorach.} Definiuje się działania:
        \begin{equation}
            \tag{Suma zbiorów}
            A \cup B = \{\,x: x \in A \vee x \in B\,\}
        \end{equation}
        \begin{equation*}
            \tag{Iloczyn zbiorów}
            A \cap B = \{\,x: x \in A \wedge x \in B\,\}
        \end{equation*}
        \begin{equation*}
            \tag{Różnica zbiorów}
            A \setminus B = \{\,x: x \in A \wedge x \notin B\,\}
        \end{equation*}
        \begin{equation*}
            \tag{Iloczyn kartezjański}
            A \times B = \{\,(a,b): a \in A, b \in B\,\}
        \end{equation*}
        
        \paragraph{Dopełnienie zbioru.}
        \begin{definition}
            Dopełnieniem zbioru $A \subset C$ do zbioru C nazywa się zbiór wszystkich elementów należących do C, które nie należą do A:
            \begin{equation*}
                \setminus A = \{\,x \in C: x \notin A\,\}
            \end{equation*}
        \end{definition}

        \paragraph{Własności dopełnienia.}
        Niech $A \subset X$, $B \subset X$. Wtedy:
        \begin{equation*}
            \setminus(\,A \cup B\,) = (\,\setminus A\,) \cap (\,\setminus B\,)
        \end{equation*}
        \begin{equation*}
            \setminus(\,A \cap B\,) = (\,\setminus A\,) \cup (\,\setminus B\,)
        \end{equation*}
        \begin{equation*}
            \setminus(\,\setminus A\,) = A
        \end{equation*}

    \subsection{Relacje}
    \begin{definition}
        Relacją nazywa się dowolny podzbiór iloczynu kartezjańskiego skończonej liczby zbiorów.
    \end{definition}

Niech R będzie relacją zadaną na $X \times X$ (tj. R jest relacją dwuargumentową, która przyjmuje za argumenty elementy X). Dodatkowo, niech $xRy$ oznacza wyrażenie "pomiędzy x a y zachodzi relacja R". Wtedy:
\paragraph{R jest relacją zwrotną} $\Leftrightarrow \forall_{x \in X} \, xRx$
\paragraph{R jest relacją przeciwzwrotną} $\Leftrightarrow \forall_{x \in X} \, \neg (\,xRx\,)$
\paragraph{R jest relacją symetryczną} $\Leftrightarrow \forall_{x,y \in X} \, (\,xRy \Longrightarrow yRx\,)$
\paragraph{R jest relacją słabo antysymetryczną} $\Leftrightarrow \forall_{x,y \in X} \, (\,xRy \wedge yRx \Longrightarrow x = y\,)$
\paragraph{R jest relacją antysymetryczną} $\Leftrightarrow \forall_{x,y \in X} \, (\,xRy \Longrightarrow \neg(\,yRx\,)\,)$
\paragraph{R jest relacją przechodnią} $\Leftrightarrow \forall_{x,y,z \in X} \, (\,xRy \wedge yRz \Longrightarrow xRz\,)$
\paragraph{R jest relacją spójną} $\Leftrightarrow \forall_{x,y \in X} \, (\,xRy \vee yRx \vee y = x\,)$
